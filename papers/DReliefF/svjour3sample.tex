%%%%%%%%%%%%%%%%%%%%%%% file template.tex %%%%%%%%%%%%%%%%%%%%%%%%%
%
% This is a template file for the LaTeX package SVJour3 for
% standardised Springer journals in twocolumn format.
%
%                                               Springer 2006/03/15
%
% Copy it to a new file with a new name and use it as the basis
% for your article. Delete % as needed.
%
%%%%%%%%%%%%%%%%%%%%%%%%%%%%%%%%%%%%%%%%%%%%%%%%%%%%%%%%%%%%%%%%%%%
%
%\documentclass[onecolumn]{svjour3}         % onecolumn (standard format)
%\documentclass[smallextended]{svjour3}     % onecolumn (second format)
\documentclass[twocolumn]{svjour3}          % twocolumn
%
\usepackage{graphicx}
\advance\paperheight by-20mm
\usepackage[cam,cross,horigin=-27.9mm,vorigin=-13.5mm]{crop}
\usepackage{color}
\usepackage{amsmath}
\usepackage{amssymb}
%
%\newcommand{\envelope}{(corresponding author)}
% if you have access to the marvosym font use its envelope symbol
\usepackage{marvosym}
\newcommand{\envelope}{(\raisebox{-.5pt}{\scalebox{1.45}{\Letter}}\kern-1.7pt)}
%
%\usepackage{mathptmx}
\usepackage{times}
\usepackage[LY1]{fontenc}
% \usepackage[LY1,mtbold]{mathtime}
% \usepackage{biograph}
%
\input{glov3.doi}
\let\SprJLogo=\relax
\DOIyear{2002}
\OFyear{2003}
\DOImsnr{6789}
\journalnumber{0815}
\journalname{Standard J.}
\idline{12: 345--678}{345}
\papertype{generic article}
%\idline{}{0}
%
\usepackage{biograph}        % to allow for author biography at the end
%
% insert here the call for the packages your document requires
%\usepackage{latexsym}
% etc.
%
% please place your own definitions here and don't use \def but
% \newcommand{}{}
%
\smartqed
% some private author definitions
\newcommand{\MSUN}{{\mathrm{MSUN}}}
\newcommand{\R}{{\mathbb{R}}}
\newcommand{\del}{\setminus}

\begin{document}

\title{Sample of the standard layout for journal
articles\thanks{Grants,
communicated-by lines, or other notes about the
article will be placed here between rules. Such notes
are optional.}}

\subtitle{A subtitle is optional. If there is one, write it here}

%\dedication{Dedicated to Professor Josef Stoer on the occasion of
%his
%60th birthday}

\titlerunning{This is the title of the article}

\author{First Author \and
%\hbox to 7.5cm{|\hrulefill\ filler \hrulefill|} \and Mechthild Stoer \and
       Second Author \and Third Author}
%\authorrunning{M. Stoer and
%       G. Dahl}


\date{Received: 1 December 2003 / Accepted:
12 February 2004 / Published online: 15 March 2004}

\institute{F. Author \and T. Author \at Konrad-Zuse-Zentrum f\"ur
Informationstechnik Berlin, Heilbronner Strasse 10, 10711 Berlin,
Germany\\\email{fauthor@zib-berlin.de} \and S. Author \envelope \at
Norwegian Telecom Research, P.O. Box 83, 2007 Kjeller, Norway\\
Tel.: +123-45-678910\columncase{,}{\\}
Fax: +123-45-678910\columncase{,}{\\}
\email{geir.dahl@tf.tele.no}}

\maketitle

\begin{abstract}
An abstract is required for regular articles. It is optional for
other types of articles, e.g., for an editorial. It is always
called ``Abstract'' in English, ``Zusammenfassung'' in German, and
``R\'esum\'e'' in French. Keywords and Mathematics subject
classification and PAC numbers are optional. Which ones are
included for a particular article is defined per journal. Lists of
nomenclature, symbols, or abbreviations are optional and only
included in some journals. Note that although all of these options
are possible, it is recommended that each journal doesn't have too
many of them. \keywords{First keyword \and Second keyword\and
More} \subclass{65K05\and 90C35}
%\ESM{Supplementary
%material is available in the online version of this article
%at \columncase{}{\\}http://dx.doi.org/10.1007/s00412-003-0244-6}
\end{abstract}

\section{Introduction}
\label{se:intro}


It is common to name the first section of an article
``Introduction''. This is, however, always optional. Headings of
the first three section levels are numbered. This introduction is an example of a
first-level heading.

The rest of this section is filler text to make the introduction a
little longer. More specifically, we have given a set $V$ of nodes
and traffic demands between certain pairs of these nodes. Each
demand represents a certain amount of point-to-point traffic to be
routed in the network between origin and destination nodes. In
addition, a set of edges joining pairs of nodes in~$V$ are given;
these represent direct physical links (for example, a fiber cable
or a radio relay system). For each edge one wants to decide which
capacity to install, chosen among a discrete set of alternatives,
each with an associated building cost. The number of such
alternatives depends on the application and the desired level of
detail. We are interested in capacity extensions such that all
demands can be routed simultaneously in the resulting network.
Such a routing is called a multicommodity flow. Especially, we may
require that the network allows a multicommodity flow also in
certain failure situations, for example, when a single edge or
node fails. In our model, we allow traffic demands to be split up
and routed on several paths, that is, we consider continuous
flows. The discreteness of the model lies in deciding the
design/capacity extension. The optimization problem in MULTISUN is
to find such a feasible network extension of minimum total
building cost.

The purpose of this paper is to present and analyze an integer linear
programming model for the MULTISUN problem using a polyhedral
approach. We study properties of polytopes that are naturally
associated with the model. Specifically, we present classes of
nonredundant inequalities that strengthen the original formulation and
may be (in fact, are) used in a cutting plane algorithm for solving
real-world planning problems. In deriving these inequalities, we
exploit relations to the knapsack problem and also the design of
(uncapacitated) networks with connectivity constraints.

\subsection{Second-level heading}
\label{level2} This is an example of a second-level heading. The
rest of the paragraph is filler text. For obtaining heat flux from
the temperature time history data of the thin film gauge, we need
the values of $E_{f}$, $\alpha$, and $\beta$. The initial voltage
calculated across each gauge, $E_f$, is  based on the initial
resistance of the gauge and the constant current fed to it;
$\beta$, the gauge backing material property, which is equal to
$(k \rho c_p)^{1/2}$ can be obtained by knowing the thermal
conductivity, mass density, and specific heat (at constant
pressure) of the backing material. The value of $\beta$ thus
obtained is 1882 J$\,$m$^{-2}$$\,$K$^{-1}$$\,$s$^{-1/2}$;
$\alpha$, the temperature coefficient of resistance of the
platinum thin film gauge is obtained from a calibration experiment
and is equal to 0.00215\,K$^{-1}$ .

\subsubsection{Third-level heading}
\label{level3} This is a third-level heading. If needed, subsequent displayed headings will be set like this one,
but will be unnumbered.

\paragraph{Paragraph heading}
This is a sample paragraph heading. It is not usually numbered.
%
\section{Sample of mathematics}
\label{se:basic_prop}

In this section, we show the standard setting of mathematics with
examples from various subject areas. In addition to what is shown
here, special formatting set by the author will be retained.

\subsection{Computer science}

Let $S$, $S_1$, $S_2$, etc.\ denote superstep processes and let
$\alpha(S)$, $\alpha_R(S)$, $\alpha_W(S)$ denote, respectively,
the set of free variables, the set of free read-only variables and
the set of free read--write variables in $S$; $x$, $y$, etc.\
denote simple variables; and $\parallel^{ss}$ denotes the
superstep composition operator. The superstep $S_1 \parallel^{ss}
S_2$ is valid provided that its constituent processes are
disjoint:
\begin{align}
\mathrm{disjoint}(S_1, S_2) \space &=_{\mathrm{df}} \space (\alpha_W(S_1)
\cap \alpha(S_2) \nonumber\\
&= \{\}) \wedge (\alpha_W(S_2) \cap \alpha(S_1) =
\{\} )
\end{align}


\begin{eqnarray}
{}^{p}\left(\bullet P\right)_{f} & = & \bullet \left(
{}^{p}P_{f}\right) \nonumber \\
{}^{p}\left(P \circ Q\right)_{f} & =
& \left({}^{p}P_{f}\right) \circ \left({}^{p}Q_{f}\right) \nonumber \\
{}^{p}\left(Ux.P\right)_{f} & = &Uk.\left( {}^{p \wedge z \neq k}\left(P^{x}_{k}\right)_{f}\right)
\ \mbox{where $k$ is a fresh name}
\nonumber \\
{}^{p}\left(x R y\right)_{f} & = & \left({}^{p}x_{f}\right) R
\left({}^{p}y_{f}\right) \nonumber \\
{}^{p}y_{f} & = & \left\{ \begin{array}{ll}
f\left(y\right) & \mbox{if $p\left(y\right) $} \nonumber \\
y & \mbox{otherwise}
\end{array} \right . \\[-4mm] \\
\nonumber{}^{p}c_{f} & = & c
\end{eqnarray}
where $x$ and $y$ are arbitrary names, $c$ is a constant, $P$ and
$Q$ are predicates, $\bullet$ is a monadic operator (such as the
logical operator, $\neg$), $\circ$ is a dyadic operator (such as
the logical operator, $\wedge$), $R$ is a relation (such as the
arithmetic relation, $<$) and $U$
is a universal quantifier (either $\forall$ or $\exists$).

\subsection{Mathematics}

For every $N\in {\bf N}$ there is ${\lambda}_N>1$ such that for
every ${\lambda}\geq {\lambda}_N$ there exists $\mu_{\lambda}>0$
such that for every $\mu\geq \mu_{\lambda}$ we have the following:
\begin{enumerate}
\item
For all $m\in{\bf N},$ for all
$\delta=(\delta_1,\dots,\delta_m)\in\{0,1\}^m$ and for all
$(k_1,\dots,k_m)\in\{1,\dots,N\}$ such that
$mn^*+k_1+\dots+k_m+\delta_1+\dots+\delta_m$ is even, there are
two $m\pi$-periodic solutions which have exactly $n^*+k_j$ zeros
in $((j-1)\pi,(j-1)\pi+\pi/2)$ and $\delta_j$ zeros in
$((j-1)\pi+\pi/2,j\pi),$ for every $j=1,\dots,m$.
\item
For any pair of sequences
$\delta=(\dots,\delta_{-1},\delta_0,\delta_1,\dots)\in\{0,1\}^{\bf
Z}$ and $k=(\dots,k_{-1},\allowbreak k_0,k_1,\dots)\in\{1,\dots,N\}^{\bf
Z}$ there are two bounded solutions  which have exactly $n^*+k_j$
zeros in $((j-1)\pi,(j-1)\pi+\pi/2)$ and $\delta_j$ zeros in
$((j-1)\pi+\pi/2,j\pi),$ for every $j\in{\bf Z}$.
\end{enumerate}


Moreover, suppose that there exist $N$ successive natural numbers
$n_1,\ldots , n_N\in {{\bf N}}$ such that
\begin{eqnarray} \label{gap}
    \dfrac{\omega \sqrt{q^+g_0}}{\pi}\leq n_1<\ldots <n_N<
      \dfrac{(b-a)\sqrt{m^+ g_\infty}}{\pi}-1
\end{eqnarray}
and let us also fix $\epsilon>0$ such that
\begin{align} \label{ipoepsilon}
    \dfrac{\omega \sqrt{q^+g_0}}{\pi}&\leq n_1<\ldots <n_N
    \columncase{}{\nonumber\\&}
    \leq \dfrac{(b-a)(1-\epsilon)\sqrt{m^+ g_\infty}}{\pi}-1.
\end{align}
Finally, let us assume that
\begin{align} \label{ipoq1}
\sqrt{m^-}&(d-c)  \columncase{>}{\nonumber\\>&}
F \left(X_0 {\mathrm{e}}^{-L \omega },
                 \sqrt{\dfrac{Hm^+}{\epsilon}}\max(1,1/\sqrt{m^+ g_\infty}) {\mathrm{e}}^{2L\omega }\right)
\end{align}
and
\begin{align} \label{ipoq2}
\sqrt{m^-}&(d-c)\columncase{>}{\nonumber\\>&}
F\left(X_0 {\mathrm{e}}^{-2L\omega },
                  \sqrt{\dfrac{Hm^+}{\epsilon}}\max(1,1/\sqrt{m^+ g_\infty}){\mathrm{e}}^{L\omega }\right),
\end{align}
where the function $F$ is defined and $L=\max(1,q^+M)$.

\subsection{Physics}

Considering the model shown above, with a transient
rate of heat transfer into the body, a general expression for the heat transfer
rate is as follows:
\begin{equation}
\dot{q}(t) = \frac{\beta}{\sqrt{\pi}}\left[ \frac{T(t)}{\sqrt{t}}
+ \frac{1}{2}\int_0^t\frac{T(t)-T(\tau)}{(t-\tau)^{3/2}}{\mathrm{d}}\tau
\right]\; . \label{eqn:D}
\end{equation}
The time dependent relative temperature $T(t)$ in the above equation can be expressed in terms of relative voltage $E(t)$, which is directly proportional to $T(t)$, and the heat transfer rate can be expressed as below:
\begin{equation}
\dot{q}(t) = \frac{\beta}{\sqrt{\pi} \alpha E_f}\left[
\frac{E(t)}{\sqrt{t}} +
\frac{1}{2}\int_0^t\frac{E(t)-E(\tau)}{(t-\tau)^{3/2}}{\mathrm{d}}\tau
\right] \label{eqn:E}
\end{equation}
where $E_f=r\cdot$I, which is a finite value, but is set as the zero level or base level for the measured voltage $E(t)$.
For a given $E(\tau)$, values of this function can be determined at
\[
 \tau = t_i = i\Delta t;\quad i=0,1,2, \ldots ,n;\quad \textrm{where},\; \Delta t = t/n.
\]
Then $E(\tau)$ can be approximated by a piece-wise linear function of the form
\begin{equation}
E(\tau) = E(t_{i-1}) + \frac{E(t_i) - E(t_{i-1})} {\Delta t}(\tau - t_{i-1}),
\label{eqn:F}
\end{equation}
where $t_{i-1} \le \tau \le t_i,\quad i=1,2,3, \ldots ,n$.


\section{Special paragraph headings}

Standard paragraph headings include theorems, propositions,
lemmas, definitions, remarks, proofs and the like. Here we show a
few examples. The text of theorems, propositions, and lemmas is
set italic. That of definitions and proofs are set upright.
Mathematics, numbers and symbols within these environments are set
the same as they are in regular text.

\begin{proposition}
\label{pr:dimension} Let $\MSUN_S$ be one of the polytopes
$\MSUN_\emptyset$, $\MSUN_E$ or finally $\MSUN_{E\cup V}$. Then
\columncase{the polytope}{}
$\MSUN_S(G,m,H,d)$ is full-dimensional if and only if
$\MSUN_S(G-e,m,H,d)$ is non\-empty for all $e\in E$.
\end{proposition}
%
\begin{proof}
Assume that $\MSUN_S(G-e,m,H,d)$ is empty for some $e\in E$.
Then either
$\MSUN_S(G,m,H,d)$ is empty,
or each feasible $x \in \R^I$ satisfies $x^1_e = 1$,
hence $\MSUN_S$ is not full-dimensional.

To prove the sufficiency, we assume that $\MSUN_S$ is not
full-dimensional, i.e., there is a linear equation $a^{\rm T}x = \alpha$
with nonzero~$a$ satisfied by each point in $\MSUN_S$.
Let $f \in E$. By hypothesis, there exists a solution~$x\in \MSUN_S$
with $x^t_f = 0$ for $t = 1, \ldots, T_{f}$.
By monotonicity, the solutions~$x^k$
obtained from~$x$ by changing $x^t_{f}$ to~1 for $t=1,\ldots,k$
are feasible. Thus we have $a^{\rm T}x = a^{\rm T}x^1 = a^{\rm T}x^2 = \cdots = \alpha$,
and by subtraction, we get $a^t_f = 0$ for all $t=1,\ldots,T_f$.
Since $f$ was chosen arbitrarily in~$E$,
we get $a = 0$, a contradiction. Thus the
equality system of $\MSUN_S$ is empty, and the polyhedron is
full-dimensional.
\qed \end{proof}

\begin{definition}
A band $B$ in~$\hat{E}$ is called a {\it valid} $\cal P${-band} if
for each nonempty $W\subset \hat{V}$ with connected shores (i.e.,
$\hat{G}[W]$ and $\hat{G}[\hat{V} \del W]$ are connected) we have
that
%
\begin{equation}
\label{eq:valid_band_def}
     m( \{\, (e,t) \in B^< \mid e \in \delta_{\hat{G}} (W)\,\} )
\;<\; d(\delta_{\hat{H}} (W) ).
\end{equation}
%
A valid $\cal P$-band~$B$ is called {\it maximal} if no valid
$\cal P$-band above~$B$ exists.
\end{definition}

\section{Other elements of an article}

\subsection{Figures and tables}

Figures and tables must always have a legend. Indicators of figure
parts are bold and lowercase. See the example in
Fig.\,\ref{chue_dgc_fig5}. An example table is shown in
Table\,\ref{tab:1}.

%\begin{figure*}
\begin{figure}
%\centering
\columncase{\includegraphics[width=.48\textwidth]{chuedf5a}\hfil}
{\includegraphics*[width=\columnwidth]{chuedf5a}\\[8pt]}
\columncase{\includegraphics[width=.48\textwidth]{chuedf5b}%
\parfillskip=0pt\endgraf}
{\includegraphics*[width=\columnwidth]{chuedf5b}}
%
\caption{Distribution of $\gamma$ (${\bf a}$) and molecular weight
(${\bf b}$) along the nozzle centerline; Mach 10 condition}
\label{chue_dgc_fig5}
\end{figure}
%\end{figure*}

% Table 1
\begin{table}
\caption{Composition parameters for states 1, 2, and 3}
\label{tab:1}       % Give a unique label
\centering
\begin{tabular}{@{}lllll@{}}
\hline\noalign{\smallskip}
$j$ & Species & $W_j$ & $\tilde n_{1,2j}$ & $\tilde n_{3j}$ \\[2pt]
\tableheadseprule\noalign{\smallskip}
1 & H$_{2} $ &  2.01588 & $2 \phi (1 - \delta)$      & 0 \\
2 & CH$_{4}$ & 16.04246 & $ \phi\delta /2$           & 0 \\
3 & O$_{2} $ & 31.9988  & 1                          & $1-\phi$ \\
4 & N$_{2} $ & 28.01348 & 3.76                       & 3.76 \\
5 & H$_{2} $O & 18.01528 & 0                          & $(2 -
\delta)
\phi$ \\
6 & CO$_{2}$ & 44.0095  & 0                          & $\phi
\delta /2$
\\[1pt]\hline
%\noalign{\smallskip}\hline
\end{tabular}
\end{table}

\subsection{Lists}

Lists can be numbered or unnumbered. When numbered, arabic
numbers, lowercase roman numerals, or lowercase letters can be
used. When unnumbered, hyphens are usually used to indicate the
start of each list item.

\begin{enumerate}
\item Solve an initial LP containing a few valid inequalities
for $\MSUN_S$, using the cost function~$c$;
\item If the optimal solution $\bar{x}$ of the current LP
is integral, output $\bar{x}$ as an optimal solution to Model~2;
otherwise $\bar{x}$ is not feasible for $\MSUN_S$, so try to find
strengthened band inequalities and other inequalities valid for
$\MSUN_S$ that are violated by~$\bar{x}$;
\item If violated inequalities were found,
add them to the current LP, solve this new LP, and go to Step~2;
otherwise stop and output $c^{\rm T}\bar{x}$
as the best lower bound found for the problem.
\end{enumerate}

\subsection{References}

A sample reference list is given at the end of this dummy article.
Three different styles for which BibTeX styles are available are
shown, using a journal and a book reference for each style. Reference
citations in the text may use numbered \cite{ref1} or author-year (Liu
2004) formats.

\section{Conclusions}

This layout is not meant to govern every last bit of content.
Obviously where an author prepares the content in an organized and
consistent manner, every effort will be made to stay true to this
presentation insofar as the required parameters of the layout
standards allow.

In addition, the instructions to authors for a particular journal
may specify additional requirements for journal style.

\begin{acknowledgements}
Acknowledgements are optional and should only be included when
provided.
\end{acknowledgements}

\appendix

\section{Appendix}
Here is an appendix. If there is more than one appendix the
heading titles should be descriptive.

\section{Appendix: proof of proposition 1}
This is an example of a descriptive title. Any numbering style for
the equations in the appendix will be accepted, provided it is
consistent and logical.

\columncase{}{\newpage}
\begin{thebibliography}{8}

\bibitem{ref1}
Manton, N.S., Speight, J.M.:
 Asymptotic interactions of critically coupled vortices.
  Commun. Math. Phys. {\bf 236}, 535--555 (2003)

\bibitem{ref2} Liu, S.: Formal Engineering for Industrial Software
Developement, pp. 302--350. Springer, Berlin Heidelberg New York (2004)

\bibitem{ref3}
M. Yousef, S.B. Qadri, E.F. Skelton, Appl. Phys. A {\bf 76}, 133
(2003)

\bibitem{ref4} C. Bernardini, L. Bonolis, eds., \emph{Enrico Fermi: His Work and
Legacy} (Springer, Berlin Heidelberg New York, 2004)

\bibitem{ref5}
Cao JY, Chen YP, Zhou ZD, Ai W (2003) Robust control of switched
reluctance motors for direct-drive robotic applications. Adv Manuf
Technol 22:184--190

\bibitem{ref6}
Craig PJ, Jenkins RO (2004) Organometallic compounds in the
environment: an overview. In: Hirner AV, Emons H (eds) Organic
metal and metalloid species in the environment. Springer, Berlin
Heidelberg New York, pp 1--15

\end{thebibliography}

\begin{authorbiography}{author.eps}{A.N. Author}\
biography will be printed here if needed. Whether an author
biography is included per author is set
per journal. A photo is optional.
\end{authorbiography}
\end{document}
